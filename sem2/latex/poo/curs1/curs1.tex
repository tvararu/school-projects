\documentclass[12pt]{extarticle}
\usepackage[utf8]{inputenc}
\usepackage{mathtools} % matrices, cases
\usepackage{amsfonts} % natural/whole/real/etc. number sets
\usepackage{geometry}
\usepackage{tikz}
\usetikzlibrary{calc}
\geometry{verbose,a4paper,tmargin=0.8cm,bmargin=0.8cm,lmargin=0.8cm,rmargin=0.8cm}
\setlength{\parindent}{0cm}

\begin{document}
	{\large
		Bibliografie: \\
		1. Herbert Schildt, C++ manual complet, Ed. Teora (sau oricare editura) \\
		2. Bruce Eckel, Thinking in C++, vol. I \\
		\\
		\underline{Limbaje de programare} \\
		Evolutie: \\
		1. Deplasarea atentiei programatorilor de la resursele hardware ale calculatorului către logica aplicatiei. \\
		2. Principalul factor care a condus la evolutia limbajelor de programare si la adaptarea unor noi limbaje pe scara largă este cresterea în complexitatea a aplicatiilor software \\
		\\
		Limbaj de asamblare = secvente de instructiuni de nivel scăzut \\
		$ \downarrow$ \\
		Limbaje de macroinstructiuni = secvente de macroinstructiuni + salturi (la alte instructiuni) \\
		$ \downarrow$ \\
		Limbaje procedurale = instructiuni repetitive, biblioteci de functii \\
		$ \downarrow$ \\
		Limbaje orientate pe obiecte = clase, obiecte \\
	}
\end{document}